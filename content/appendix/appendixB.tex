%===================================== APPENDIX B =================================
\chapter{User Manual}

\textbf{Note:} This appendix explains how to user the web interface of the visualization tool. Remember that some of the functionality require that you have added the available callbacks to your Keras code. We will not explain how to do this in this appendix, instead we refer the user to the appendix C for the callbacks API as well as an explanation of how to add them to your code. \\

\noindent The web interface is fairly simple to use. At all times, the top of the page will show a navigation bar with links to the available pages. When clicked, these links take you to the corresponding page. The interface will display error feedback for any illegal action you might try to take, such as trying to log in with a mismatching username and password, or trying to upload a file with a filename that is not unique. Similarly, feedback for successful activity, like uploading a file, is also shown. \\

% Login page

\noindent The first time you start the visualization tool, you will be shown a login page. If you do not already have a user, you can press the "Create user" link to create a new user for the application. Usernames must be between 5 and 20 characters, and can only contain alphanumerical characters and underscores. The username must also be unique. The password must be longer than 8 characters and can only contain alphanumeric characters. When you have successfully created a user, you are automatically redirected back to the login page, where you can log in to start using the visualization tool. \\

% create user

\noindent When you have logged in, you will be directed to a page showing all your uploaded files. If there are now uploaded files yet, the page will suggest that you click on a link to upload a file. If you already have files uploaded, these will be displayed in a list with the columns: filename, tags, the uploaded date, and the date that the file was last ran. If the file is currently running, the last column will show a yellow label that reads "Running". You can sort the list by clicking on the various columns. A search bar is also provided that responds to both the filename and tags. Clicking on a row in the list will take you to the corresponding file's overview page. But first, we will describe the upload page. \\

% Uploaded files

\noindent You can upload files directly from your computer on the file upload page. The file uploaded will be given the actual filename, so make sure this is a describing name. If you want to upload several versions of the same file, you will need to change the filename between each upload since the application requires unique filenames. For better organizing your files, you can associate tags to the file before uploading. The tags are split at whitespaces, for instance "neural network" will result in two tags: "neural" and "network". Use underscores if you want to add tags containing multiple words. The filename, without its extension, will be automatically added as a tag. \\

% upload file

\noindent When a file is successfully uploaded, you are taken to the file's overview page. A submenu will be available with seven tabs: the overview, the file output, the training progress, and the rest of the visualization techniques. The overview page shows some information about the file, as well as the contents of the file. You have the option of downloading the generated network, assuming that you have added the network saving callback and ran the file before. You can also delete the file if you wish. Finally, you can start running the file by pressing the large green button that reads "Run". Upon clicking the run button, a popup window will appear and ask you to select an image to use for the visualizations. If you have previously ran the file, the previously selected image will be available. Upload a new one, or use the previous image by not selecting a new image, and press "Run". Note that if the file already has been ran, it will have data associated with it, for instance networks or visualization data files. Running the file again will delete all of this data, and the popup will contain a warning of this. Finally, if you press the run button, the code in the file will be ran as a Python subprocess. When a file is running, it will have the yellow "Running" label displayed at the top right of its overview page. You can also kill the execution of the file at any time by clicking the stop button. A banner with a link to the file will be shown whenever one of your files is done running.\\

% File overview page
% Run popup
% File is running?

\noindent While a file is running, you can navigate to the various tabs to see different output and visualizations from the file. The output tab shows the last 50 lines of the command line output of the file in real-time while its running. This can be useful for instance if your file unexpectedly stops because of some error, or if you just want to see the ETA of your training. You can download the full output by clicking the button. \\

% Output tab

\noindent \textbf{Note: }The next tabs described require that you have added the corresponding callbacks to your code. \\ % something about how often they are updated

\noindent The training progress tab naturally shows the progress of the training. Two plots are available: one for the accuracy and one for the loss. The data shown is the accuracy and loss at each batch in the training (unlike the command line output which shows the average accuracy and loss for the current epoch) over the epoch share. This will give you an overview of how the accuracy and loss are changing through the training process. If you have enabled validation in your Keras code, the validation accuracy and validation loss will be displayed on the graph at each full epoch as a green, circular point. You can hover over these points to see the exact value if you have the hover tool selected on the toolbar. Other tools available are zooming and panning the plot, as well as saving the plot and resetting to the default view. \\

% Training progress

\noindent The next tab shows the layer activations. Note that this tab may take some time to load, depending on how many layers your network have. Once loaded, you can see the input image, followed by the activations for the different layers. Input, dropout and flatten layers are automatically excluded if not specified otherwise in the callback.  For convolutional and pooling layers, the activations of the filters are concatenated into grids separated by white borders. The dense layers are displayed in  \\% how to explain this??? Also need to write why layer activations are useful.

% Layer activations

\noindent We can view the saliency map of the uploaded visualization image on the next tab. 

% explain saliency maps, deconvolution network and deep visualization

\cleardoublepage