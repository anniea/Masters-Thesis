%===================================== APPENDIX D =================================
\chapter{Visualization Files} \label{app:visualization-files}

This appendix presents the format of the visualization files produced from the callbacks. The information can be useful for developers that want to extend the application or work directly with the visualization files instead of through the provided interface. \\

\noindent All examples shown are generated using a grayscale input image of shape (28, 28, 1). All arrays must be NumPy arrays.

\section{Training Progress}

\textbf{Note:} \texttt{progress} refers to the percentage of the training progress that the metrics occur at, for instance 0.1 is 10\% of the first epoch, 1.5 is halfway in the second epoch, and so on.

\begin{itemize}
    \item \textbf{Filename:} training\_progress.txt
    \item \textbf{Format:} \texttt{progress loss accuracy}
    \item \textbf{Example:}
\end{itemize}

\begin{minted}
[
bgcolor=mygray,
fontsize=\footnotesize
]
{text}
0.0 0.109375 2.315384864807129
0.0021321961620469083 0.140625 2.287837028503418
0.0042643923240938165 0.234375 2.2661919593811035
0.006396588486140725 0.1875 2.2656376361846924
0.008528784648187633 0.2890625 2.199150562286377
0.010660980810234541 0.25 2.2080719470977783
\end{minted}

\noindent If validation is enabled, a separate text file for validation is created:

\begin{itemize}
    \item \textbf{Filename:} training\_progress\_val.txt
    \item \textbf{Format:} \texttt{progress val\_loss val\_accuracy}
    \item \textbf{Example:}
\end{itemize}

\begin{minted}
[
bgcolor=mygray,
fontsize=\footnotesize
]
{text}
1 0.97000 0.09730
2 0.97850 0.06922
\end{minted}


\section{Layer Activations}

\begin{itemize}
    \item \textbf{Filename:} layer\_activations.pickle
    \item \textbf{Format:} \texttt{[(layer\_name, array), ...]}
    \item \textbf{Example:}
\end{itemize}

\begin{minted}
[
bgcolor=mygray,
fontsize=\footnotesize
]
{text}
[('Layer 1: conv2d_1', array(shape=(32, 26, 26), dtype=uint8), 
 ('Layer 2: conv2d_2', array(shape=(32, 24, 24), dtype=uint8),
 ('Layer 3: max_pooling2d_1', array(shape=(32, 12, 12), dtype=uint8),
 ('Layer 6: dense_1', array(shape=(128,), dtype=uint8),
 ('Layer 8: dense_2', array(shape=(64,), dtype=uint8),
 ('Layer 9: dense_3', array(shape=(10,), dtype=uint8)]
\end{minted}

\section{Saliency Maps}

\begin{itemize}
    \item \textbf{Filename:} saliency\_maps.pickle
    \item \textbf{Format:} \texttt{array}
    \item \textbf{Example:}
\end{itemize}

\begin{minted}
[
bgcolor=mygray,
fontsize=\footnotesize
]
{text}
array(shape=(28, 28, 1), dtype=uint8)
\end{minted}

\section{Deconvolution Network}

\begin{itemize}
    \item \textbf{Filename:} deconvolution\_network.pickle
    \item \textbf{Format:} \texttt{[(array, layer\_name, feat\_map\_no), ...]}
    \item \textbf{Example:}
\end{itemize}

\begin{minted}
[
bgcolor=mygray,
fontsize=\footnotesize
]
{text}
[(array(shape=(28, 28, 1), dtype=uint8), 'max_pooling2d_1', 0),
 (array(shape=(28, 28, 1), dtype=uint8), 'max_pooling2d_1', 10),
 (array(shape=(28, 28, 1), dtype=uint8), 'max_pooling2d_1', 22),
 (array(shape=(28, 28, 1), dtype=uint8), 'max_pooling2d_1', 3)]

\end{minted}

\section{Deep Visualization}

\begin{itemize}
    \item \textbf{Filename:} deep\_visualization.pickle
    \item \textbf{Format:} \texttt{[(array, layer\_name, unit\_index, loss\_value), ...]}
    \item \textbf{Example:}
\end{itemize}

\begin{minted}
[
bgcolor=mygray,
fontsize=\footnotesize
]
{text}
[(array(shape=(28, 28, 1), dtype=uint8), 'dense_3', 8, 11761.4814453125),
 (array(shape=(28, 28, 1), dtype=uint8), 'dense_3', 9, 7111.62060546875),
 (array(shape=(28, 28, 1), dtype=uint8), 'dense_2', 8, 3238.24609375),
 (array(shape=(28, 28, 1), dtype=uint8), 'dense_1', 112, 0.0),
 (array(shape=(28, 28, 1), dtype=uint8), 'conv2d_2', (12, 12, 0), 0.0),
 (array(shape=(28, 28, 1), dtype=uint8), 'conv2d_2', (12, 12, 8), 0.0)]
\end{minted}
