\clearpage
\pagenumbering{roman} 				
\setcounter{page}{1}

\pagestyle{fancy}
\fancyhf{}
\renewcommand{\chaptermark}[1]{\markboth{\chaptername\ \thechapter.\ #1}{}}
\renewcommand{\sectionmark}[1]{\markright{\thesection\ #1}}
\renewcommand{\headrulewidth}{0.1ex}
\renewcommand{\footrulewidth}{0.1ex}
\fancyfoot[LE,RO]{\thepage}
\fancypagestyle{plain}{\fancyhf{}\fancyfoot[LE,RO]{\thepage}\renewcommand{\headrulewidth}{0ex}}

\section*{\Huge Sammendrag}
\addcontentsline{toc}{chapter}{Sammendrag}	
$\\[0.5cm]$

\noindent Datamaskiner blir stadig kraftigere, og kunstige nevrale nettverk er igjen et populært forskningsfelt. Økningen i dypde og kompleksitet kompliserer imidlertid forskeres evne til å forstå hvordan et slikt nettverk oppfører seg. Som en følge av dette har en mengde visualiseringsteknikker blitt utviklet, med mål om å forbedre forståelsen av nevrale nettverk. Gjennom visualisering kan forskere få innsikt i hva som foregår inne i nettverkene sine. Bildene kan gjøre identifisering av svakheter lettere, som videre kan føre til forslag om forbedringer. Dessverre finnes det få alternativer for å forenkle bruken av slike visualiseringer. Denne oppgaven presenterer et verktøy som visualiserer treningsdataen et nettverk produserer, ved hjelp av etablerte visualiseringsteknikker. Hovedformålet med verktøyet er å tilgjengeliggjøre nyttige visualiseringer som kan bidra til å forstå nevrale nettverk, gjennom et praktisk API og webgrensesnitt. \\

\noindent I denne oppgaven har vi studert en rekke visualiseringsteknikker for implementering i verktøyet. Vi har også gjennomgått eksisterende visualiseringsverktøy for ulike maskinlæringsbiblioteker, for å finne mangler i utvalget. For å utvide utvalget har vi implementert et visualiseringsverktøy for maskinlæringsbiblioteket Keras, som kombinerer administrasjon av nettverk med avanserte visualiseringer i sanntid. For å demonstrere verktøyets evner, har vi inkludert og tolket visualiseringer produsert fra to eksempelnettverk. \\

\noindent I tillegg til det implementerte visualiseringsverktøyet presenterer denne oppgaven også en casestudie i ansiktsgjenkjenning. Teknologien for ansiktgjenkjenning utvikler seg stadig til det bedre, men støter fortsatt på problemer i møte med den virkelig verden. Variasjoner i bildekvalitet, positur, belysning og ansiktsuttrykk byr på store utfordringer. Selv om nevrale nettverk har hatt stor suksess innen ansiktsgjenkjenning, er det fortsatt rom for forbedring. \\

\noindent Oppgaven presenterer en ny metode for å forbedre systemer for ansiktgjenkjenning ved bruk av nevrale nettverk, som utnytter informasjonen i ansiktsuttrykk istedenfor å prøve å overkomme dem. Casestudiet undersøker tre separate tilnærminger for nettverksarkitektur: en uten bruk av ansiktsuttrykk, en med en ekstra inngangsnode for ansiktsuttrykk, og en med en ekstra utgangsnode for ansiktsuttrykk. Ved å sammenlikne resultatene, så vi at nettverkene med en ekstra inngangsnode gjorde det bedre enn tilsvarende standardnettverk, mens nettverkene med en ekstra utgangsnode alt i alt gjorde det dårligere enn tilsvarende standardnettverk. Dette indikerer at den foreslåtte tilnærmingen med en ekstra inngangsnode er interessant for videre forskning.

\clearpage