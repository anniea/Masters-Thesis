\clearpage
\pagenumbering{roman} 				
\setcounter{page}{1}

\pagestyle{fancy}
\fancyhf{}
\renewcommand{\chaptermark}[1]{\markboth{\chaptername\ \thechapter.\ #1}{}}
\renewcommand{\sectionmark}[1]{\markright{\thesection\ #1}}
\renewcommand{\headrulewidth}{0.1ex}
\renewcommand{\footrulewidth}{0.1ex}
\fancyfoot[LE,RO]{\thepage}
\fancypagestyle{plain}{\fancyhf{}\fancyfoot[LE,RO]{\thepage}\renewcommand{\headrulewidth}{0ex}}

\section*{\Huge Abstract}
\addcontentsline{toc}{chapter}{Abstract}	
$\\[0.5cm]$

\noindent With the recent advancement in computational power, artificial neural networks are once again on the rise. However, the increase in depth and complexity further complicates researchers' ability to make sense of a networks' behaviour. To that end, a number of visualization techniques have been developed, aiming to improve the understanding of artificial neural networks. Through visualization, researchers can gain insights into the inner workings of their networks, displaying millions of parameter values in one simple image. These images can allow for easier identification of limitations and weaknesses, that can further lead to suggestions on improvements. Unfortunately, there exists a limited number of options to facilitate the creation of such visualizations. This thesis presents a visualization tool that utilizes data produced by a network during training, to create visualizations using well-established visualization techniques. The main objective of the tool is to provide easy access to useful visualizations that can aid in improving and understanding artificial neural networks, through a convenient API and web interface. \\

\noindent In this thesis, we have studied a number of visualization techniques to be implemented in our tool. We have also reviewed existing visualization tools for various machine learning libraries, to identify gaps in the selection. To improve on the selection, we have implemented a visualization tool for Keras that combines the management of networks with advanced, real-time visualizations. By testing the visualization tool on two example networks and interpreting the visualizations produced, we have demonstrated the tool's ability to facilitate a deeper understanding of artificial neural networks and their behaviour. \\

\noindent In addition to the implemented visualization tool, the thesis also presents a case study in face recognition. The area of face recognition research continues to push boundaries, but still face difficulties with recognition in unconstrained environments. Obstacles like image quality, pose, illumination, and expressions present many challenges. The application of artificial neural networks provide an opportunity to create face recognition systems that excel in this area.\\

\noindent The thesis presents a novel method for improving face recognition systems using artificial neural networks, that utilizes information available in facial expressions instead of trying to overcome expressions through invariance. Nine experimental networks were created based on three separate approaches: no utilization of expression information, an additional expression input, and an additional expression output. Examining their performance showed that the networks with an additional expressions input performed better than the baseline network. However, the networks with an additional expression output performed worse than the baseline network. The findings indicate that the proposed approach would be interesting to research further.

\clearpage