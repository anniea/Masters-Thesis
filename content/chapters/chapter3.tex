%===================================== CHAP 3 =================================

\chapter{Implementation}

\textit{Detailed description of the web interface and its structure, this was missing from the specialization project. \\
\noindent Present the technology used (might have to reword the name of this chapter). \\
\noindent Include several typical software development diagrams.}


\section{Design}

\subsection{System Development Methodology}

The development of the visualization tool has been carried out by a team consisting of the two of us, with no specific set of roles. Because of the small team size, we did not see the need for a dedicated project manager. Both of us were involved in designing the architecture and interface, as well as implementing and testing the system, giving us great knowledge of the whole system. Still, we maintained an efficient workflow by dividing the programming task into two main responsibility areas, one for each of us:
\begin{enumerate}
    \item Visualizing data using the selected visualization library and incorporating these into the user interface
    \item Implementing visualization techniques and creating networks to illustrate the use of the tool with the chosen deep learning library
\end{enumerate}

\noindent The development process itself did not follow any strict guidelines, but included several elements from Agile methods. It has been an iterative and incremental process, starting with the existing prototype of the system that only implemented two of the visualization techniques in a very simple manner. Based on our own testing as well as feedback from the customer, which in this case is our supervisors, functionality were added and adjustments were made in order to obtain a new and improved prototype. This process was repeated until we reached a satisfactory system according to the initial requirements. An example of the iterative part of the process is when we decided to replace the old visualization library with a new one that better suited our needs. An example of the incremental part of the process is when our supervisors expressed the wish for a way for the user to be able to upload an image to be used in producing some of the visualizations. Other concepts from Agile development that were put into use are code review, pair programming, and the use of a backlog to get an overview of the requirements.

\subsection{Focus Quality Attributes}

\subsubsection{Modifiability}

An important aspect of the system design of the visualization tool is to make sure that any part can be easily replaced. For instance, it should not be too difficult to alter the system to use a different deep learning library than the one employed in our implementation. This requires thorough consideration when designing the architecture, and typically calls for a module-based architecture with each module being loosely coupled, meaning that it should interact with as few of the other modules as possible.

\subsubsection{Usability}

Not only should the interface be simple and easy to navigate for the user, but the actual installation of the program and the adaption of the user's deep learning scripts to the tool, should be without much trouble. A thorough user manual and documentation of the API is the key to obtaining this kind of usability. Preferably, we would want any ANN to run in our program without problems, but it is near to impossible to generalize this much. The main goal was thus to support the most commonly used networks, but at the same time, as the previous section mentions, easily allow for extending the system to work with different kinds of ANNs.

\subsection{Technology Decisions}

\subsubsection{Flask}

Flask is a web framework written in Python, used to build simple web applications. It is a micro-framework, meaning that it has very few dependencies to external libraries. It provides you with only the basic tools needed to create a web application, and is therefore very lightweight. There exists many plugins that can be added for an increased range of functionality. Since the focus of our visualization tool is not the web page itself, but rather its capabilities in terms of visualization techniques, Flask is the perfect choice. It allowed us to quickly set up an application and implement a basic user system and upload functionality.

\subsubsection{SQLite}

Very lightweight and easy to include in simple desktop apps that could make use of a database. Do not really need to store much data. Simple relational data. Not an important part of the application. Lightweight. Not a lot of connections.

\subsubsection{Keras}

Keras is a high-level neural networks library that can run on top of either TensorFlow or Theano. Its purpose is to provide a way to quickly and easily create models and start experimenting. It is very minimalistic, providing only enough to achieve an outcome, while still allowing for extensibility by making it easy to add and use new modules within the framework. Keras provides the user with several callbacks, i.e. sets of functions to be applied at giving stages of the ANN training procedure. It also allows for creating your own custom callbacks, which is perfect for our use case, since we then can create callbacks that produces the data needed for each visualization technique.

\subsubsection{TensorFlow}

Not sure if we need this section. But mention in Keras section, and say whether we support both Theano and TensorFlow?

\subsubsection{Bokeh}

Bokeh is a Python interactive visualization library for modern web browsers. It can easily be embedded into a Flask application, as we will show in a later section. The library is easy to use, and has the flexibility of adding interactions and highly advanced customization. Another benefit is that it is easy to stream large data sets and plot them live. Since Bokeh is a fairly new library that is still under major development,


there are some drawbacks in that it is rapidly changing and 

It has the flexibility to add interactions, and can stream large data sets live.

A lot of possibilities for adding interactions, easy to stream live plotting of data. Drawbacks are that it is fairly new and still under development. Also probably not the best for handling images.

\section{Architecture}

\subsection{Overview}
% First, the overall architecture. Explain the different modules.
% Then we can focus on some parts of the architecture that we want to "show off".

\begin{figure}
    \centering
        \includegraphics[width=1\textwidth]{fig/overall-architecture.pdf}
        \caption{Overall Architecture of the System}
        \label{fig1}
\end{figure}

\subsection{Spotlight1}

\subsection{Spotlight2}


% I am not sure where the following should be placed:

% Something about what improvements can be done, probably in further work section.

\begin{itemize}
    \item Focus was not on efficiency
\end{itemize}

\section{Existing Solutions}

\subsection{TensorBoard}

\subsection{DIGITS}

\subsection{Yosinski Deep Vis Toolbox}

\section{Visualization Techniques} 
% Perhaps in a result section?
% Should possibly include something about them in implementation as well
% ex. this is implemented after this article, but with some modifications etc.

\subsection{Training Progress}

\subsection{Layer Activations}

\subsection{Saliency Maps}

\subsection{Deconvolution Network}

\subsection{Deep Visualization}

\section{Callbacks}
% Just say something about HOW the visualizations are implemented.
% We can refer the reader to an Appendix on exactly how to use them and what they are.


\cleardoublepage