%===================================== CHAP 6 =================================

\chapter{Discussion}

This chapter will discuss the results presented in the previous chapter.

\section{Visualization Tool}

% Interpret and explain your results
% Answer your research question
% Justify your approach
% Critically evaluate your study

\subsection{The MNIST Network}


\subsubsection{Training Progress}

\subsubsection{Layer Activations}

\begin{comment}
FIG. 1.5:
- We see that the first convolutional layer do not find any specific features
- It is not until the second convolutional layer that we see features in terms of edges.
- Fc 3: the network is 100\% certain that it is class 0.

FIG. 1.6:
- Shows conv2 at 0, 5 and 10. Should have 1 as well.
\end{comment}

\subsubsection{Saliency Maps}

\subsubsection{Deconvolutional Network}

\subsubsection{Deep Visualization}

\subsection{The VGG Network}

\subsubsection{Training Progress}

\subsubsection{Layer Activations}

\subsubsection{Saliency Maps}

\subsubsection{Deconvolutional Network}

\subsubsection{Deep Visualization}

\subsection{Discussion} % kalle denne noe annet

% lettere å se når du sammenlikner

\section{Case Study in Face Recognition}


% for extra output: loss is high, but accuracy isnt THAT bad

% The combined metrics of the extra output architecture make it harder to compare to the baseline and extra input architecture.

% The extra output architecture underperforms in comparison with the other two, but still manages an impressive accuracy considering it is tackling two problems simultaneously.

\cleardoublepage